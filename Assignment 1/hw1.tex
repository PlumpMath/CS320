\documentclass[11pt]{article}
\usepackage{fullpage}
\usepackage{graphics,epsfig,color}
\usepackage{wrapfig}

\usepackage{times}
\usepackage{setspace}
\usepackage{amsmath,amsthm,amssymb}
%\usepackage[ruled,vlined,linesnumbered]{algorithm2e}
\usepackage{qtree}
\usepackage{subfigure}
\usepackage{url}

\usepackage{fancyhdr}
\pagestyle{fancy}


%for code from latexdraw
%\usepackage[usenames,dvipsnames]{pstricks}
%\usepackage{epsfig}
%\usepackage{pst-grad} % For gradients
%\usepackage{pst-plot} % For axes


\newtheorem{theorem}{Theorem}[section]
%\newtheorem{proposition}{Proposition}[theorem]
\newtheorem{corollary}{Corollary}[section]
\newtheorem{lemma}{Lemma}[section]
%\newtheorem{claim}{Claim}[section]
\newtheorem{problem}{Problem}
%\newtheorem{conjecture}{Conjecture}[section]
\newtheorem{definition}{Definition}[section]
\newtheorem{observation}{Observation}[section]
\newtheorem{example}{Example}[section]
\newtheorem{openproblem}{Open Problem}[section]
\newtheorem{fact}{Fact}[section]
%\newcommand{\qedsymb}{\hfill{\rule{2mm}{2mm}}}

\newcommand{\qedsymb}{\hfill{\rule{2mm}{2mm}}}
\newenvironment{proofsketch}{\begin{trivlist}
\item[\hspace{\labelsep}{\noindent Proof Sketch: }]
}{\qedsymb\end{trivlist}}



%the following few lines until usepackage{algorithm2e} is to avoid the
%conflicts of algorithm2e with other packages.
\makeatletter
\newif\if@restonecol
\makeatother
\let\algorithm\relax
\let\endalgorithm\relax
%\usepackage[ruled,vlined,linesnumbered]{algorithm2e}
\usepackage[ruled,vlined,linesnumbered]{algorithm2e}


%\newenvironment{proof}{\begin{trivlist}
%\item[\hspace{\labelsep}{\bf\noindent Proof: }]}{\qedsymb\end{trivlist}}
%\newcommand{\qed}{\hfill\rule{2mm}{2mm}}

\newcommand{\remove}[1]{}



%--------------------------------


\begin{document}

%\thispagestyle{empty}
\renewcommand{\headrulewidth}{0.4pt}
%\thispagestyle{fancy}
%\lhead{\bf CSCD320 Exam 1, Fall 2011, EWU\\Name: \hspace*{5cm}EWU Email: \hspace*{5cm}EWUID:}
%\chead{}
%\rhead{Submitted to ISSTA'04}
%\lfoot{}
%\cfoot{\it Submitted to}
%\rfoot{}

\fancyhead[L]{\bf CSCD320 Homework1, Winter 2012, 
Eastern Washington University. Cheney, Washington. \\
\bigskip Name: Eric Fode\hspace{40mm}EWU ID:005301214}


\noindent
\rule[0.1cm]{16.5cm}{0.01cm} 





\vspace*{2mm}

\newpage

\begin{problem}[5 points]
\label{prob:1}
  Based on your learning from the CSCD300 Data Structures course,
  describe your understanding of the connection and difference between
  the ``data structures'' and ``algorithms''.  Say your opinions in
  your own language. Any reasonable opinion is welcome.
\end{problem}


%---------------------------------------

\begin{problem}[15 points]
\label{prob:2}
  Show: $3n^2+n\sqrt{n}=O(n^2)$
\end{problem}



%---------------------------------------

\begin{problem}[15 points]
\label{prob:3}
Show: $2(n+100\sqrt{n})\log^2 n = o(n\sqrt{n}/\log n)$. 
{\em (Note: $\log^2 n$ means $(\log n)^2$)}
\end{problem}




\begin{problem}[15 points]
\label{prob:4}
  Let $f(n)$ and $g(n)$ be asymptotically nonnegative functions.  Show:
  $f(n)+g(n) = \Theta(\max\{f(n), g(n)\})$, using the definition of
  $\Theta$.
\end{problem}



%---------------------------------------


\begin{problem}[15 points]
\label{prob:5}
Let $f(n)$ be an asymptotically nonnegative function.
Is $f(n)=\omega(f(\sqrt{n}))$ always true ? Justify your answer. 
\end{problem}



%---------------------------------------
\begin{problem}[20]
\label{prob:6}
  Finding the maximum in a sequence of $n$ numbers is a simple task:
  simply scan the sequence and pick/return the maximum. The time cost
  of this simple method is clearly $\Theta(n)$, which is optimal
  because one has to take one look at every number in the sequence in
  order to find the maximum.
  Now Dr.\ Nonsense wants to use the divide-conquer strategy to
  overkill this task of finding the maximum among the sequence of $n$
  number.  
\begin{enumerate}
\item 
Describe the algorithmic idea and give the pseudocode of Dr.\
Nonsense's algorithm.
\item Give the time cost in the $\Theta$-notation of Dr.\ Nonsense's
  algorithm by using the recursion tree method for solving the
  recurrence. 
\item Is this Dr.\ Nonsense's d-c based algorithm asymptotically slower
  or faster or identical, compared with the simple $\Theta(n)$-time
  method ? 
\item How is the time efficiency in practice of Dr.\ Nonsense's
  algorithm, compared the simple method ? Why ? 
\end{enumerate}
 
\end{problem}



%---------------------------------------


\begin{problem}[15 points total; 5 points for each algorithm.]
\label{prob:7}
 Search and learn three existing algorithms that use the
  divide-conquer strategy (not those covered in the class and/or the
  textbook, such as merge sort, max subarray finding algorithm, and
  Strassen's matrix multiplication algorithm). For each algorithm, in
  your own language, concisely and clearly describe:

\begin{enumerate}
\item  the problem statement

\item  the algorithmic idea in the solution (don't just copy the code to me)

\item the time complexity

\item the condition, on which the worst-case running time appears. 

\item the source of your finding. For example, the url of the webpages, the 
title and page of a book, the title/author/year of an article, etc. 
\end{enumerate}
\end{problem}



%\end{document}

\newpage

%---------------------------------------

\bigskip
\noindent{\bf Solution for Problem~\ref{prob:1}.}
I really don't know ....

If you want to use itemize ....
\begin{itemize}
\item aaaaa. 
\item bbbbb.
\end{itemize}

%I am commented out !!

If you want to use enumerate ...

\begin{enumerate}
\item ccccc
\item dddddd
\end{enumerate}

%---------------------------------------
\bigskip

\noindent{\bf Solution for Problem~\ref{prob:2}.}


\begin{proof}
This problem has no sense, so there is no proof, but have to create
something
as constants $c$ and $n_0$,
  such that when $n\geq n_0$,
$$
3n^2\log^2 n+n\sqrt{n}+\log n + \frac{n^2}{n^5}\leq c n^2
$$

That is, 
$
3n^2\log n+n\sqrt{n}+\log n + \frac{n^2}{n^5}\leq c n^2
$
\end{proof}

%---------------------------------------
\bigskip

\noindent{\bf Solution for Problem~\ref{prob:3}.}


blaalala, i have a picture to show .... in Figure~\ref{fig:test}.

\begin{figure}[h!]
\begin{center}
\includegraphics[scale=0.1]{Fig-6-4.pdf}
\caption{test figure}
\label{fig:test}
\end{center}
\end{figure}

%---------------------------------------
\bigskip

\noindent{\bf Solution for Problem~\ref{prob:4}.}

\begin{proof}
find your own solution ... for 
 
$$c_1[\max\{f(n), g(n)\}]\leq f(n)+g(n) \leq c_2[\max\{f(n), g(n)\}] 
\textrm{, for any }
n\geq n_0$$

blalalala

\end{proof}

%---------------------------------------
\bigskip

\noindent{\bf Solution for Problem~\ref{prob:5}.}

\begin{proof}
I don't know. below are insane. 
$f(n)=\omega(f(\sqrt{n}))$ ?

\begin{itemize}
\item
yes ?  if $f(n)$, then
$f(\sqrt{n})\neq \omega((1/2)\log n)$. 
\item
no ?  if $f(n)=\infty$, then
$\sqrt{n}$. Clearly, 
$\omega(n)$.
\end{itemize}
\end{proof}

%---------------------------------------
\bigskip

\noindent{\bf Solution for Problem~\ref{prob:6}.}


{\bf Idea:} find it ! \\



{\bf Pseudocode:}  Below uses the algroithm2e package (.sty file must
be in your working directory), but you don't have to use it. It's just
for a nicer output ... 


\begin{algorithm}[H]
%\SetAlgoLined
\NoCaptionOfAlgo
%\SetAlgoNoLine
\DontPrintSemicolon
\caption{\bf max($A, p, r$)} 

\lIf{$1>2$}{Report insane. Exit.}
\bigskip

\lIf{$x=y$}{\Return{$z$}}.

\bigskip

$x \leftarrow \lfloor(9\cdot G)/45\rfloor$
\bigskip

$x \leftarrow \max(A, p, q)$

$y \leftarrow \min(A, q+1, r)$
\bigskip

\end{algorithm}

google search for algorithm2e, you will find documents on how to use
the macros given by algorithm2e.\\


\newpage

You can also use the {\bf following}. 

\begin{verbatim}
dafda
dafdafdafda
  daljdlk 
 jkalf jalj lk jlg jlkajg klg ;jsfg jsf 
j4l26j5i6,v5bb b53bu 63n j6b j
\end{verbatim}

The {\em verbatim} macro will tell latex to output exactly what you
type, so it's a convienent way to let the output to be what you see as
what you have. So using verbatim might be another way to write your
pseudo code, but the bad thing is you cannot type those ``greek''
letters ......  

%---------------------------------------
\bigskip

\noindent{\bf Solution for Problem~\ref{prob:7}.}

Sorry, i have no idea about this problem. 

\vspace*{+1cm}

{\LARGE 
Play around with this sample, try different tricks/options/toys to see
the change in the output. Whenever you have a question, it's alwyas a
good idea to ask Google. You can also google search for
docs/videos/slides that teach how to use latex. 
}



Last note about including a picture. 

\begin{verbatim}
You can save a figure in a pdf, jgg, or eps files, and use the command
\includegraphics{figfilename} to includge the figure (find the example
in the sample article). 

If you use Linux/Mac and use command line to complile you latex file
to get the output pdf file. 

1. if your figure file is in .pdf or .jpg, you have to use the command

> pdflatex hw1.tex

2. if your figure file is in a .eps file, then you have to type
commands:

> latex hw1.tex
> dvipdf hw1.dvi


or (if you also want to get the hw.ps file as the by-product)

> latex hw1.tex
> dvips hw1.dvi
> ps2pdf hw1.ps

\end{verbatim}

\newpage
About how to create a figure file, there are many tool for that. One
thing you should know is that you figure should have a tight bounding
box, meaning no wide empty area is left around the figure, so that
you won't have space wasted in your article.\\


Several example tools to create figs. \\

1. use ppt to create the figure, print that particular slide into a
pdf file, open the pdf using FULL acrobat (acrobat reader does not
have this function), crop the figure are out (I remeber ``crop''
should be somewhere in the ``too'' menu item in the acrobat).  Then
save the cropped figure out as a new pdf file or eps file, depending
on your taste. \\

2. several other tools on linux: xfig, latexdraw. I use latexdraw a
lot for my lecture slides. \\

3. you also find other tools that I don't know from google search.  




\end{document}




