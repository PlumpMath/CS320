\documentclass[11pt]{article}
\usepackage{fullpage}
\usepackage{graphics,epsfig,color}
\usepackage{wrapfig}
\usepackage{times}
\usepackage{setspace}
\usepackage{amsmath,amsthm,amssymb}
\usepackage{url}
\usepackage{fancyhdr}
\pagestyle{fancy}


\newtheorem{theorem}{Theorem}[section]
\newtheorem{corollary}{Corollary}[section]
\newtheorem{lemma}{Lemma}[section]
\newtheorem{problem}{Problem}
\newtheorem{definition}{Definition}[section]
\newtheorem{observation}{Observation}[section]
\newtheorem{example}{Example}[section]
\newtheorem{openproblem}{Open Problem}[section]
\newtheorem{fact}{Fact}[section]

\newcommand{\qedsymb}{\hfill{\rule{2mm}{2mm}}}
\newenvironment{proofsketch}
{
	\begin{trivlist}
	\item[\hspace{\labelsep}{\noindent Proof Sketch: }]
}{\qedsymb\end{trivlist}}



%the following few lines until usepackage{algorithm2e} is to avoid the
%conflicts of algorithm2e with other packages.
\makeatletter
\newif\if@restonecol
\makeatother
\let\algorithm\relax
\let\endalgorithm\relax
\usepackage[ruled,vlined,linesnumbered]{algorithm2e}

\newcommand{\remove}[1]{}



%--------------------------------


\begin{document}

	\renewcommand{\headrulewidth}{0.4pt}
	
	\fancyhead[L]{\bf CSCD320 Homework1, Winter 2012, 
	Eastern Washington University. Cheney, Washington. \\
	\bigskip Name: Eric Fode\hspace{40mm}EWU ID:005301214}
	
	
	\noindent
	\rule[0.1cm]{16.5cm}{0.01cm} 
	
	\vspace*{2mm}
	
	

	
	
	\bigskip
	\bigskip
	\noindent{\bf Solution for Problem 1}
	The diffrence between data structures and algorithms is the difference 
	between form and function. Data structures gives your data form, 
	it determines how you store your data. Algorithms determine how you work
	with your data. They are related concepts in that all algorithms operate
	on some data structure even it is as simple as a list. 
	%---------------------------------------
	\bigskip
	
	
	\noindent{\bf Solution for Problem 2}
	
	
	\begin{proof} 
		Show that $3n^2 + n\sqrt{n} = O(n^2)$
		
		$$\displaystyle \lim_{n \to \infty}\frac{ 3n^2 + n\sqrt{n}}{n^2} = 3$$
		
		\noindent 
		so
		
		 $$3n^2 = O(n^2)$$
	
	\end{proof}
	
	%---------------------------------------
	\bigskip
	
	\noindent{\bf Solution for Problem  3}
	
	\begin{proof}
		Show that $\displaystyle 2(n + 100\sqrt{n})(log(n))^2 = o(\frac{n\sqrt{n}}{log(n)})$
		
		
		$$\displaystyle \lim_{n \to \infty}\frac{2(n + 100\sqrt{n})(log(n))^2}{\frac{n\sqrt{n}}{log(n)}} = 0$$
		so
		$$\displaystyle 2(n + 100\sqrt{n})(log(n))^2 = o(\frac{n\sqrt{n}}{log(n)})$$
	\end{proof}
	
	
	
	\newpage
	\noindent{\bf Solution for Problem 4}
	
	\begin{proof}
		it will always be true that $f(n) \leq g(n)$ or that $g(n) \leq f(n)$
		so we can say that  
		$$2(\max\{f(n),g(n)\}) \geq f(n) + g(n)$$
		givin that defenition of $\theta$ implies that 
		$$0 < \lim_{n \to \infty} \frac {f(n)}{g(n)} < \infty $$
		if
		$$\lim_{n \to \infty}\frac{2\max\{f(n),g(n)\}}{\max\{f(n), g(n)\}} = 2$$
		then
		$$\lim_{n \to \infty}\frac{f(n) + g(n)}{\max\{f(n), g(n)\}} \leq 2$$
		because
		$$\frac{2\max\{f(n),g(n)\}}{\max\{f(n),g(n)\}} \geq \frac{f(n) + g(n)}{\max\{f(n),g(n)\}}$$
		a
		$$\lim_{n \to \infty} \frac{f(n) + g(n)}{\max\{f(n), g(n)\}} > 0 $$
		because
		$$\lim_{n \to \infty} \frac{f(n) + g(n)}{f(n)} > 0 \textrm{ and } \lim_{n \to \ \infty} \frac{f(n) + g(n)}{g(n)} > 0$$
		so by definition $f(n) + g(n) = \theta(\max\{f(n), g(n)\})$
	
	\end{proof}
	
	%---------------------------------------
	\bigskip
	
	\noindent{\bf Solution for Problem 5}
	
	\begin{proof}
		is $f(n)=\omega(f(\sqrt{n}))$ always true?\\
		$f(n)= \omega(f(\sqrt{n}))$ is shown by
		$$\displaystyle \lim_{n \to \infty}\frac{n}{\sqrt{n}} = \infty$$
		so as long as $n_0 > 0$ it is true $f(n) = \omega(f(\sqrt{n}))$
	
	\end{proof}
	
	%---------------------------------------
	\newpage
	
	\noindent{\bf Solution for Problem 6}
	
	{\bf 1. Pseudocode:}
	
	\begin{algorithm}[H]
		\NoCaptionOfAlgo
		\caption{\bf max($a, lower, upper$)}
		
		\KwIn{A Section of an array $a$ with bounds at $lower$ and $upper$}
		\KwResult{The largest number in $a$ will be returned}
		\Begin{
				\If{$upper \neq lower$}{
					$mid \longleftarrow ((upper - lower) / 2) + lower$\;
					\Return $\max\{max(a,lower,mid),max(a,mid+1,upper)\}$\;
				}
				\Else{
					\Return $lower$\;
				}
		}
	\end{algorithm}
	{\bf 2. $\theta$ analysis using tree}
	The recurance for that discribes this algroithm is
	$$T(N)=2T(N/2) + 1$$
	There will be $log(n)$ layers to the tree\\
	The sum of each layer will be $2^d$ where d is the depth of the layer//
	So the bottom in terms of n will be 
	\begin{align*}
	2^{log_{2}n} &=\\
	n^{(log_{2})} &=\\
	&=n
	\end{align*}
	the summation is
	$$2^0+2^1+2^2+ ... +n$$
	$$\sum\limits_{i=0}^{log{2}(n)} 2^i = 2n-1$$
	so O is 
	$$O(n)$$
	{\bf 3. Asymptotic Compairson}\\
	Asymptotically the algoritims are the same\\
	{\bf 4. Reality Compairson}\\
	In practice the d-c algorithm will be half as fast as the simple method
	because it is doing twice as many comparisons
	
	%-------------------------------------------------------
	\newpage
	
	\noindent{\bf Solution for Problem 7}
  \begin{enumerate}
		\item {\bf Towers of Hanoi}
		The problem is divided smaller steps that consist of moving the nth ring to the middle peg then
		to the far right peg, n is incresed from the smallest ring to the largest until all of the rings
		have been moved
		\item {\bf Multiplication of numbers}
		For a givin $x$ and $y$, $x$ and $y$  is broken into two parts $x_h$ and $x_l$ where the high 
		and low parts are each half of the number represented as an array then the algorithim recureses 
		with the paramaters a $x$ array and a $y$ (four times total) when the arrays reach a size of one
		they are multiplyed to together then combied with bit shifts.
		\item {\bf Finding nth greatest number}
		This algorithim spilts the array in to three parts,a greater, less then, and equal to
		then some number (in the array) then counts the number of items in each part
		and then if the count of items in the less then array is less then 
		then n (the place of the number a sorted version of the array) then it recureses and
		runs the algorithim again on the less then array, if it is greater then the less then
		count it adds the count of equals items to the count if n is now less then the count
		then it recurses on to the equals array with n = n - count of items in the less then array
		. If none of these are true then it recurses onto the greater then array with n = n- the 
		count of all items not in the greater then array.
	\end{enumerate}
		
	
	
\end{document}




