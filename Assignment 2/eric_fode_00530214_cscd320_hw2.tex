\documentclass[11pt]{article}
\usepackage{fullpage}
\usepackage{graphics,epsfig,color}
\usepackage{wrapfig}
\usepackage{times}
\usepackage{setspace}
\usepackage{amsmath,amsthm,amssymb}
\usepackage{url}
\usepackage{fancyhdr}
\usepackage{enumitem}
\pagestyle{fancy}


\newtheorem{theorem}{Theorem}[section]
\newtheorem{corollary}{Corollary}[section]
\newtheorem{lemma}{Lemma}[section]
\newtheorem{problem}{Problem}
\newtheorem{definition}{Definition}[section]
\newtheorem{observation}{Observation}[section]
\newtheorem{example}{Example}[section]
\newtheorem{openproblem}{Open Problem}[section]
\newtheorem{fact}{Fact}[section]

\newcommand{\qedsymb}{\hfill{\rule{2mm}{2mm}}}
\newenvironment{proofsketch}
{
	\begin{trivlist}
	\item[\hspace{\labelsep}{\noindent Proof Sketch: }]
}{\qedsymb\end{trivlist}}



%the following few lines until usepackage{algorithm2e} is to avoid the
%conflicts of algorithm2e with other packages.
\makeatletter
\newif\if@restonecol
\makeatother
\let\algorithm\relax
\let\endalgorithm\relax
\usepackage[ruled,vlined,linesnumbered]{algorithm2e}

\newcommand{\remove}[1]{}



%--------------------------------


\begin{document}

	\renewcommand{\headrulewidth}{0.4pt}
	\setlength{\headheight}{38.0pt}
	\fancyhead[L]{\bf CSCD320 Homework1, Winter 2012, 
	Eastern Washington University. Cheney, Washington. \\
	\bigskip Name: Eric Fode\hspace{40mm}EWU ID:005301214}


	\noindent{\bf Solution for Problem 1}
	%---------------------------------------
	\begin{proof} Prove that $T(n) = 2T(n/3) +n^2 = \theta(n^2)$ using induction\\
	 	{\bf Claim:} if $T(n) = 2T(n/3) +n^2$ then $T(n) \leq c(n^2)$ for some constant c\\
	 	{\bf Assume:}	
	 	$$T(1) = 2$$
	 	$$c = 2$$
		Base Case of 3:	
		 	\begin{align*}
		 		T(3) &= 2T(3/3) + 3^2\\
		 		&=2 * 2 + 9\\
		 		&= 13
		 	\end{align*}
		Hypothesis: assume $T(m) \leq c*m^2$ is true for all $m = 3,...,n$\\
		Inductive step: when $m = n + 1$
		\begin{align*}
			T(n+1) &= 2T(n/3) + n^2\\
			&=2T(\frac{n+1}{3}) + (n+1)^2\\
			&<= c (n+1)^2 + (n+1)^2\\
			&< (n+1)^2(c+1)
		\end{align*}
	\end{proof}
	\newpage
	\noindent{\bf Solution for Problem 2}
	\begin{enumerate}
		\item $T(n) = 4T(n/2) + 3n^2 - 9n$\\
			$a = 4, b =2, f(n) = 3n^2 -9n$
			case 2
			$$3n^2 - 9n = \theta(n^2)$$
			so
			$$T(n) = \theta(n^2)$$
		\item $T(n) = 4T(n2) + 2n^3 - 100 n^2$\\
			$a = 4, b =2,f(n) = 2n^3 - 100 n^2$	case 3
			$$2n^3 - 100n^2 = \Omega(n^{2+1})$$
			and
			$$4 * ((n) ^ 3 - 50n^2) \leq 1/2 (2n^3 - 100 n^2) $$
			so
			$$T(n) = \theta(n^3)$$
		\item $T(n) = 4T(n/2) + n + 5 log n$\\
			$a = 4 , b= 2, f(n) = n + 5 log n$ case 1
			$$n + 5 \log n = O(n^{2 -1})$$
			so
			$$T(n) = \theta(n^2)$$
		\item $T(n) = 8T(n/2) + n^2 + n log n$\\
			$a = 8 , b= 2, f(n) = n^2 + n log n$ case 1
			$$n^2 + n \log n = O(n^{3 -1})$$
			so
			$$T(n) = \theta(n^3)$$
		\item $T(n) = 8T(n/2) + 4n^3 + 5n^2$\\
			$a = 8 , b= 2, f(n) = 4n^3 + 5n^2$ case 2
			$$4n^3 + 5n^2 = \theta(n^{3})$$
			so
			$$T(n) = \theta(n^3\log n)$$
			
		\item $T(n) = 8T(n/2) + 2^{-10} n^4 + 6n^3$\\
			$a = 8 , b= 2, f(n) = 2^{-10} n^4 + 6n^3$	case 3
			$$ 2^{-10} n^4 + 6n^3 = \Omega(n^{3+1})$$
			and
			$$ \frac{2^{-10} n^4}{2} + 3n^3 \leq c (2^{-10} n^4 + 6n^3)$$
			when n is large
			so
			$$T(n) = \theta(n^4)$$	
	\end{enumerate}
	%---------------------------------------
	\bigskip
	
	\noindent{\bf Solution for Problem  3}
	
	
	\bigskip
	\noindent{\bf Solution for Problem 4}
	\begin{enumerate}[leftmargin=0pt]
	\item Idea: navigate to bottom of tree and step back one level
		compair subtree left max, subtree right max and this tree total return max of these three
		repeate until top node is reached
		\item  Pseudocode:
		
		\begin{algorithm}[H]
				\NoCaptionOfAlgo
				\caption{\bf maxtree($tree$)}
				\KwIn{A tree}
				\KwResult{The root node of the largest subtree}
				\Begin{
							\If{$tree == null$}
							{
								return 0\;
							}
							\If{$tree.leaf == true$}
							{
								$tree.head.value = tree.head.weight$\;
								return $tree.head$\;
							}
							$leftReturn \longleftarrow maxtree(tree.left)$\;
							$rightReturn \longleftarrow maxtree(tree.right)$\;
							$tree.head.value \longleftarrow tree.left.value + tree.right.value + tree.head.weight$\;
							
							\If{$leftReturn.value > rightReturn.value$ {\bf and} $leftReturn.value > tree.head.value$}
							{
								\Return $leftReturn$\;
							}
							\If{$rightReturn.value > leftReturn.value$ {\bf and} $rightReturn.value > tree.head.value$}
							{
								\Return $rightReturn$\;
							}
							\If{$tree.head.value > leftReturn.value$ {\bf and} $tree.head.value > rightReturn.value$}
							{
								\Return $tree.head$\;
							}
							\If{$leftReturn.value == rightReturn.value$ {\bf and}\\ $rightReturn.value > tree.head.value$}
							{
								\Return $rightReturn$\;
							}
							
						}
				
			\end{algorithm}
		\newpage
		\item Analyze:
			This algorithm can be represented by the following recurance 
			$$T(n) = 2T(n/2)+ c $$
			which using the master therom can be converted to
			$$T(n) = O(n)$$
	\end{enumerate}
	%---------------------------------------
	\bigskip
	
	\noindent{\bf Solution for Problem 5}\\
	{\bf BST}\\
	{\bf Pros:}
	\begin{enumerate}
		\item extermly space effcent, they only take up $O(n*c)$ space.
		\item fast lookup, access time for a givin index is $O(\log_{2} n)$.
		\item fast find, find time is $O(\log_{2} n)$ assuming a compair operation is defined.
		\item assuming a compair operation is defined items are sorted as they are inserted.
		\item if converted to anouther type of data structure it is trivial to retain order of items.
	\end{enumerate}
	{\bf Cons:}
	\begin{enumerate}
		\item Insert can be and cause tree to have to restructure.
		\item Remove can be expensive and cause tree to restructure.
		\item Insert and remove are not constant time operations.
	\end{enumerate}
	{\bf Best Use Cases:}
	\begin{enumerate}
		\item When data will not be added or deleted frequently.
		\item When searching is required.
		\item When only one ordering will be used
	\end{enumerate}
	{\bf Hash Table}\\
	{\bf Pros:}
	\begin{enumerate}
		\item Extermly fast lookup $O(1)$.
		\item Extermly fast insert (assuming that hash table is large enugh to hold new item) $O(c)$.
		\item Extermly fast delete $O(1)$.
		\item Extermly fast modify $O(c)$.
	\end{enumerate}
	{\bf Cons:}
	\begin{enumerate}
		\item Slow search $O(n)$.
		\item No ordering.
		\item List can be very large to accomidate room for all possible hashs.
	\end{enumerate}
	{\bf Best Use Cases:}
	\begin{enumerate}
		\item When data is constantly being added and removed.
		\item When data is ordered on hash.
		\item When space is unlimited and speed is critical
	\end{enumerate}
	
\end{document}




